\section{Symmetrien in der Physik}

\subsection{Aufbau der Materie}

In dieser Vorlesung werden wir uns mit den elementaren Bausteinen der Materie
und ihren Wechselwirkungen befassen. Dadurch soll ein quantifizierbarer und
nachvollziehbarer Zugang zur Wirklichkeit gefunden werden. Zunächst sollten wir
jedoch die Frage klären, was ``elementar'' überhaupt bedeutet. In der
Vergangenheit wurden zunächst die Atome für unteilbar gehalten, dann entdeckte
man Elektronen, Neutronen, Protonen und heute sprechen wir von ``Quarks'' als
Elementarteilchen. Im Folgenden werden wir uns mit diesem Postulat kritisch
auseinandersetzen.
\begin{figure}[!ht]
  \centering
\begin{pspicture}(-0.2,-4.4)(7.5,1.25)

\rput(1.5,0.7){Elementarteilchen}
\rput(5.2,0.7){Wechselwirkungen}

\psellipse[linewidth=0.04,dimen=outer](1.5,0.7)(1.5,0.5)
\psellipse[linewidth=0.04,dimen=outer](5.2,0.7)(1.5,0.5)

\rput(3.2,-0.17){Hadronen}
\psline{->}(3.2,-0.47)(3.2,-0.87)

\rput(3.2,-1.17){Kerne}
\psline{->}(3.2,-1.47)(3.2,-1.87)

\rput(3.2,-2.17){Atome}
\psline{->}(3.2,-2.47)(3.2,-2.87)

\rput(3.2,-3.17){Moleküle}
\psline{->}(3.2,-3.47)(3.2,-3.87)
\psline{->}(4.2,-3.47)(4.8,-3.87)
\psline{->}(2.2,-3.47)(1.6,-3.87)

\rput(3.2,-4.17){Flüssigkeit}
\rput(5.2,-4.17){Festkörper}
\rput(1.2,-4.17){Gas}

% Trennlinien
\psline(5.15,-1.6)(7.4,-1.6)
\psline(1.55,-2.8)(0.0,-2.8)

\psline{->}(5.2,0)(5.2,-1.4)
\psline{->}(1.5,0)(1.5,-2.6)

\psline{->}(5.2,-1.8)(5.2,-3.87)

\rput[l](5.4,-0.8){Starke und}
\rput[l](5.4,-1.2){Schwache WW}
\rput[l](5.4,-2.6){EM WW}
\rput{-90}(0.6,-0.4){\rput(0.9,0.4){Diese Vorlesung}}
\end{pspicture} 
  \caption{Übersicht Elementarteilchen, Wechselwirkungen.}
\end{figure}

Primär werden wir es mit der starken, schwachen sowie der elektromagnetischen
Wechselwirkung zu tun haben. Die Gravitation wird in dieser Vorlesung nur eine
untergeordnete Rolle spielen und kann so lange wir Elementarteilchen
betrachten vernachlässigt werden.

\sfigure[H]%
	{1-EigenschaftenDerWechselwirkungen.pdf}
	{\BethgeSchroeder, S. 10}
	{Charakteristische Eigenschaften der Wechselwirkungen.}
	
\sfigure[H]%
	{1-Langenskalen.pdf}
	{\FrauenfelderHenley, S. 2}
	{Typische Längenskalen. Der Bereich unterhalb von $10^{-18}\mathrm{m}$ ist 
		unerforscht! Es ist nicht bekannt, ob dort neue Kräfte und Erscheinungen zu
		erwarten sind.}%
	
\sfigure[H]%
	{1-Anregungsenergien.pdf}
	{\FrauenfelderHenley, S. 2}
	{Bereich der Anregungsenergien. Die angegebenen Temperaturen entsprechen den
	jeweiligen Energien.}%
	
\sfigure[H]%
	{1-Dichtebereiche.pdf}
	{\FrauenfelderHenley, S. 2}
	{Dichtebereiche.}%
	
\subsubsection{Symmetriebrechung}

Es sei eine Temperaturskala mit Phasenübergang bei einer kritischen Temperatur
$T_C$ gegeben. In der Thermodynamik verwenden wir die \emph{freie Energie} $F$,
um den Zustand eines Systems zu beschreiben, wobei die freie Energie von einem
Ordnungsparameter $\ph$ abhängt.

Generell erhält man für hohe Temperaturen bzw. Energien eine höhere Symmetrie
welche bei niedrigeren Temperaturen gebrochen wird.
\begin{figure}[!ht]
  \centering
\begin{pspicture}(-1,-1)(4.5,2.5)
 \psaxes[labels=none,ticks=none]{->}%
 (0,0)(-0.5,-0.5)(4,2)[,-90][\color{gdarkgray}$T$,0]
 
 \psyTick(1){\color{gdarkgray}T_C}
 
 \rput[l](0.5,1.5){Hohe Symmetrie}
 
 \rput[l](0.5,0.5){Niedere Symmetrie}
\end{pspicture}
  \caption{Symmetriebrechung durch Temperaturänderung.}
\end{figure}

\begin{bspn}
\begin{figure}[!ht]
  \centering
\begin{pspicture}(-2.5,-1)(2.5,2.5)
 \psaxes[labels=none,ticks=none]{->}%
 (0,0)(-2,-0.5)(2,2)[\color{gdarkgray}$\ph$,-90][\color{gdarkgray}$F$,0]

 \pscircle[linewidth=0.6pt](0,0.2){0.2}
 
 \psplot[linewidth=1.2pt,%
	     linecolor=darkblue,%
	     algebraic=true]%
	     {-1.5}{1.5}{0.6*x^2}
 
 \rput[l](1.2,0.4){\color{gdarkgray}$T>T_c$}
 
 \rput(0,-0.8){\color{gdarkgray}Symmetrie}
\end{pspicture}
\begin{pspicture}(-2.5,-1.6)(2.5,2.5)
 \psaxes[labels=none,ticks=none]{->}%
 (0,0)(-2,-0.5)(2,2)[\color{gdarkgray}$\ph$,-90][\color{gdarkgray}$F$,0]
 
 \pscircle[linewidth=0.6pt](0.85,-0.35){0.2}
 
 \psplot[linewidth=1.2pt,%
	     linecolor=darkblue,%
	     algebraic=true]%
	     {-1.5}{1.5}{x^4-1.5*x^2}
 
 \rput[l](0.2,1.4){\color{gdarkgray}$T<T_c$}
 
 \psbezier[arrows=<->](-0.85,-0.6)(-0.55,-1.2)(0.55,-1.2)(0.85,-0.6)
 \rput(0,-1.4){\color{gdarkgray}Symmetriebrechnung}
\end{pspicture}
  \caption{Symmetriebrechung in der Thermodynamik.}
\end{figure}

Wir kennen dieses Verhalten bereits von Ferromagneten. Hier wird die Ordnung
durch den Ordnungsparameter Magnetisierung ($\ph=M$) beschrieben. Wird eine
kritische Temperatur überschritten (\emph{Curie-Temperatur}), so haben die
elementaren Dipole keine Vorzugsrichtung mehr und der Magnet wird zum
Paramagneten.
\begin{figure}[!ht]
  \centering
\begin{pspicture}(-1,-2)(4,2.5)
 \psaxes[labels=none,ticks=none]{->}%
 (0,0)(-0.5,-2)(3.5,2)[\color{gdarkgray}$H$,-90][\color{gdarkgray}$T$,0]

 \psplot[linewidth=1.2pt,%
	     linecolor=darkblue,%
	     algebraic=true]%
	     {0}{1.15}{x^4}
 
 \psplot[linewidth=1.2pt,%
	     linecolor=darkblue,%
	     algebraic=true]%
	     {0}{1.15}{-x^4}
 
  \rput[bl](0.2,1.2){\color{gdarkgray}para}
 \rput[bl](0.2,-1.2){\color{gdarkgray}ferro}
\end{pspicture}
\begin{pspicture}(-1,-1)(4,2.5)
 \psaxes[labels=none,ticks=none]{->}%
 (0,0)(-0.5,-0.5)(3.5,2)[\color{gdarkgray}$T$,-90][\color{gdarkgray}$M$,0]
 
 \psarc[linewidth=1.2pt,%
	    linecolor=darkblue]%
	   (0.5,0){2}{0}{87}
	   
 \psxTick(2.5){\color{gdarkgray}T_c}
 
 \rput[lb](1.5,0.25){\color{gdarkgray}ferro}
 \rput[lb](2.7,0.25){\color{gdarkgray}para}
\end{pspicture}
  \caption{Verhalten eines Ferromagneten bei Curie-Temperatur.}
\end{figure}

Bei dieser Temperatur liegt eine höhere Symmetrie vor, denn der Magnetismus ist
isotrop also Richtungsunabhängig. Kühlt man den Magneten ab, ordnen sich die
Dipole wieder entlang ihrer Vorzugsrichtung starr an, die räumliche Symmetrie
innerhalb der ferromagnetischen Domäne geht verloren. Diesen Vorgang nennt man
spontane Symmetriebrechung.

Betrachten wir die Welt um uns herum, so finden wir zahlreiche gebrochene
Symmetrien, die bei hohen Energien bzw. Temperaturen wiederhergestellt werden.
Kristalle schmelzen; gebundene Zustände werden in ihre Bestandteile zerlegt wie
z.B. Gase gehen in Plasmen über etc.

Um die elementaren Bestandteile der Materie zu finden, ist es daher hilfreich,
die Energie immer weiter zu erhöhen und die verbleibenden Symmetrien zu
skechieren.
\bsphere
\end{bspn}

Zur Beschreibung der Gravitation verfügen wir über die Einsteinschen
Feldgleichungen und die Allgemeine Relativitätstheorie, für den
Elektromagnetismus über die Maxwell Gleichungen. In beiden Fällen
sind die Gleichungen lorentzinvariant, d.h. Lorentztransformationen sind
elementare Symmetrieoperationen. 
Für die starke und schwache Wechselwirkung sind (noch) keine Grundgleichungen
verfügbar, man ist daher auf der Suche nach zusätzlichen Symmetrien.
 
Das Konzept der Symmetrie spielt eine wichtige Rolle bei hohen Energien, da
dort neue Symmetrien auftreten. Man vermutet, dass sich bei sehr hohen Energien
die starke, schwache und elektromagnetische Wechselwirkung und die Gravitation
zu einer einzigen Wechselwirkung mit einer neuen Symmetrie vereinen. Diese
Symmetrie wird im Niederenergiebereich gebrochen und führt zu den
unterschiedlichen Wechselwirkungen, die wir mit niederer Symmetrie heute
beobachten.
\twofigures[!htbp][0.4][0.5]%
	{1-VergleichGesamtwirkungsquerschnitte.pdf}%
	{\FrauenfelderHenley, S. 428}%
	{1-EnergienVergleich.pdf}%
	{P. A. Tipler, R. A. Llewellyn, Modern Physics, Fifth Edition S. 605}
	{Vergleich der Gesamtwirkungsquerschnitte von hadronischen, elektromagnetischen
	und schwachen Prozessen an Nukleonen. $\sigma$ bezeichnet den geometrischen
	Wirkungsquerschnitt eines Nukleons, $K$ gibt die kinetische Energie an.}%

Die vermutetete Symmetrie kann helfen, die Zustände zu
erklären, die bei der Entstehung des Universums vorzufinden waren. Denn zu
jeder neuen Symmetrie gehört auch ein neuer Erhaltungssatz. Aber auch das
Verhalten komplexer Kernmaterie zeigt Phasenübergänge und Symmetriebrechung. 

\sfigure[H][0.8]%
	{1-PhasendiagrammKerne.pdf}%
	{\FrauenfelderHenley, S. 514}%
	{Phasendiagramm für Kerne.}


\begin{bspn}
Betrachten wir die Temperaturkurve von Kernmaterie in Abhängigkeit der
Anregungsenergie pro Nukleon, so ergibt sich eine große Ähnlichkeit zum Verlauf der Temperaturkurve von
Wasser in Abhängigkeit der Anregungsenergie pro Molekül. Solche Analogien
helfen und dabei, das Verhalten der Kerne auf bekanntes zu übertragen und so
ein Modell für die Kerne zu entwickeln.\bsphere
\end{bspn}

\sfigure[H]%
	{1-TemperaturdiagrammKernmatWasser.pdf}%
	{\BethgeWalter, S. 216}%
	{Temperaturdiagramm für (a) Kernmaterie und (b) Wasser.}

\subsection{Symmetrien und Erhaltungssätze (klassisch)}

In diesem Abschnitt sollen Symmetrien mathematisch formuliert werden. Durch
diesen Formalismus lässt sich aus jeder Symmetrie ein Erhaltungssatz ableiten.

Unter einer Symmetrie versteht man eine Operation $\op{S}$, die ein
physikalisches Objekt oder Gesetzt $O$ invariant lässt,
\begin{align*}
\op{S}(O) = O.
\end{align*}
$O$ nennt man auch \emph{Observable}.

\begin{bspn}
\emph{Paritätssymmetrie}. 
Wir beobachten ein Phänomen bspw. eine Teilchenbewegung sowohl gewöhnlich als
auch durch einen Spiegel. Eine Symmetrie liegt genau dann vor, wenn sich die
Physik im Spiegelbild nicht ändert.\bsphere
\end{bspn}
\begin{bspn}
\emph{Translationssymmetrie}.
Wechseln zwei Teilchen in einem abgeschlossenen homogenen System miteinander,
so ist die Wechselwirkung vom Koordinatenursprung unabhängig.
\begin{figure}[!ht]
  \centering
\begin{pspicture}(-0.1,-1.79)(5.6,1.77)
\pscircle[linewidth=0.04,dimen=outer](0.3,0.24){0.32}
\pscircle[linewidth=0.04,dimen=outer](5.2,1.38){0.32}

\psline(0.7928483,0.4704805)(1.3136523,0.21520971)(1.695397,0.7339302)(2.2282233,0.44050884)(2.5597954,0.9853575)(3.0855687,0.64777446)(3.4432688,1.2427961)(3.9439557,0.9182771)(4.2624636,1.4380393)(4.674827,1.1276263)
\psline(3.5,-1.7)(0.74,-0.09)
\psline(3.5,-1.7)(5.16,0.87)
\psdots(3.5,-1.7)

\rput(0.3,0.24){\color{gdarkgray}1}
\rput(5.2,1.38){\color{gdarkgray}2}

\rput(4.9,-0.2){\color{gdarkgray}$\vec{x}_1$}
\rput(2,-1.25){\color{gdarkgray}$\vec{x}_2$}
\rput(2.4,1.4){\color{gdarkgray}$V(\abs{\vec{x}_1-\vec{x}_2})$}
\end{pspicture} 
  \caption{2-Teilchen-Translationsinvarianz.}
\end{figure}

Eine Symmetrieoperation ist daher gegeben durch,
\begin{align*}
\hat{S}: \vec{x}_i \mapsto \vec{x}_i + \vec{a},
\end{align*}
wobei $\vec{a}$ für einen beliebigen Koordinatenursprung steht.
$\op{S}$ verändert als Symmetrieoperation die Wechselwirkung $V$
zwischen den Teilchen nicht, d.h. $\hat{S}(V) = V$. Aufgrund dieser Symmetrie erhalten
wir sofort die \emph{Impulserhaltung}.
\begin{align*}
\vec{p} = \vec{p}_1+\vec{p}_2,
\end{align*}
denn es gilt
\begin{align*}
\frac{\diffd}{\dt}\vec{p} = -\nabla V = -(\nabla_1+\nabla_2)V = -\nabla_1V -
\nabla_2V = 0.\Rightarrow \vec{p}=\const.\bsphere
\end{align*}
\end{bspn}
Um das letzte Beispiel zu verallgemeinern, führen wir die
\emph{Hamiltonfunktion}
\begin{align*}
H(p_i,q_i),
\end{align*}
ein mit den generalisierten Koordinaten $q_i$ und den kanonischen Impulsen
$p_i$. Eine Translation hat nun die Form,
\begin{align*}
\hat{S}: q_i \mapsto q_i + \delta q_i.
\end{align*}
Gilt $\hat{S}(H) = H$, so ist $\hat{S}$ eine Symmetrieoperation, d.h. $\hat{S}$
verändert die Hamiltonfunktion nicht. In diesem Fall gilt
\begin{align*}
&0 = \delta H = \delta q\sum_i \frac{\partial H}{\partial q_i} = -\delta q\sum_i
\dot{p}_i.\\
\Rightarrow & \frac{\diffd}{\dt} \sum_i \dot{p}_i = 0.
\end{align*}
Wir erhalten somit aus der Translationssymmetrie den Impulserhaltungssatz.

Ananlog erhält man aus der Rotationssymmetrie den Drehimpulserhaltungssatz und
aus der zeitlichen Translationssymmetrie die Energieerhaltung.

\subsection{Symmetrien in der Quantenmechanik}
Wir wollen die in der klassischen Mechanik entwickelte Theorie nun auf die
Quantenmechanik erweitern.
\begin{enumerate}[label=\arabic{*}.)]
\item Die verallgemeinerten Impulse und Koordinaten werden durch Operatoren
  ersetzt, die einer Vertauschungsrelation genügen.
  \begin{align*}
  \nrm{\hat{p}_i,\hat{q}_k} = i\hbar\delta_{ik}.
  \end{align*}
\item Die zu messenden Größen werden durch Erwartungswerte ersetzt,
\begin{align*}
\lin{\hat{\Omega}} = \int \dV \Psi^* \hat{\Omega}\Psi.
\end{align*}
\end{enumerate}
Ist $\hat{\Omega}$ eine Erhaltungsgröße, verschwindet die zeitliche Ableitung.
Es gilt 
\begin{align*}
i\hbar \frac{\diffd}{\dt}\lin{\hat{\Omega}}
&= \int\dV i\hbar \frac{\diffd}{\dt}\left[\Psi^*\hat{\Omega}\Psi\right]
\\&= \int \dV \left[\left(-\Psi^*i\hbar\frac{\diffd}{\dt}\right)\hat{\Omega}\Psi
 + \Psi^*\hat{\Omega}\left(i\hbar\frac{\diffd}{\dt}\Psi\right)\right]
\\ &= \int\dV -(\Psi^*\hat{H})\hat{\Omega}\Psi + \Psi^*\hat{\Omega}H\Psi
\\ &= \int\dV \Psi^*\left(\hat{\Omega}\hat{H} - \hat{H}\Omega\right)\Psi
\\ &= \lin{\nrm{\hat{H},\hat{\Omega}}}. 
\end{align*}
$\op{\Omega}$ ist also genau dann eine Erhaltungsgröße wenn $\op{H}$ und
$\op{\Omega}$ vertauschen.
\begin{bspn}
Drehung um die $z$-Achse.
\begin{align*}
\hat{S}:\; & x\mapsto x\cos\ph - y\sin\ph,\\
& y\mapsto x\sin\ph + y\cos\ph,\\
&z\mapsto z.
\end{align*}
\begin{figure}[H]
\centering
\begin{pspicture}(0,-1.35)(1.4178126,1.35)
\psline{->}(0.21,-0.50453126)(0.24,1.0649999)
\psline{->}(0.21,-0.50453126)(1.06,0.025)
\psline{->}(0.21,-0.50453126)(1.0,-1.0550001)

\rput(1.2878126,0.235){$y$}

\rput(1.1654688,-1.205){$x$}

\rput(0.2334375,1.255){$z$}
\psellipse[linecolor=darkblue](0.23,0.56000006)(0.23,0.17)
\psline[linecolor=darkblue]{->}(0.38,0.45000005)(0.46,0.51000005)
\end{pspicture} 
\caption{Drehung um die $z$-Achse.}
\end{figure}
Für eine Drehung um $\delta \ph$ gilt daher,
\begin{align*}
&\delta x = - y\delta\ph,\\
&\delta y = x\delta\ph,\\
&\delta z = 0,\\
\Rightarrow&\hat{S}\Psi(\vec{x}) = \Psi(\vec{x}) - \delta x
\frac{\partial\Psi}{\partial x} + \delta y \frac{\partial\Psi}{\partial y}
= \left(1-\delta\ph\left(x\frac{\partial}{\partial y}-y\frac{\partial}{\partial
x} \right) \right)\Psi(\vec{x}).
\end{align*}
Die Drehung hängt also mit der $z$-Koordinate des Drehimpulses
zusammen,
\begin{align*}
\hat{l}_z = \nrm{\hat{x}\times\hat{p}}_z = -i\hbar
\left(x\frac{\partial}{\partial y} - y\frac{\partial}{\partial x}\right).
\end{align*}
Die Drehung um die $z$-Achse können wir damit schreiben als,
\begin{align*}
\op{S}\Psi(\vec{x}) = \left(1- \frac{i}{\hbar}\partial\ph
\op{l}_z\right)\Psi(\vec{x}).
\end{align*}
Da wir infinitesimale Winkel $\delta\ph$ betrachten, können wir dies als
Taylor-Entwicklung 1. Ordnung des Exponential-Operators auffassen. Beim
Übergang zur kontinuierlichen Drehung erhält man,
\begin{align*}
\hat{S}:\;\Psi(\vec{x})\mapsto
\exp\left\{\frac{i}{\hbar}\ph\hat{l}_z\right\}\Psi(\vec{x}).
\end{align*}

Man bezeichnet $\hat{l}_z$ als \emph{Erzeugende} der Drehung um
die $z$-Achse. Betrachten wir nun den Hamilton Operator unter einer infinitesimalen
Drehung,
\begin{align*}
\hat{S}\left(\lin{\hat{H}}\right)
&= \int\dV \Psi^*\left(1-\frac{i}{\hbar}\delta\ph \op{l}_z\right)\hat{H}
\left(1+\frac{i}{\hbar}\delta\ph \op{l}_z\right)\Psi\\
&= \int\dV \Psi^*\hat{H}\Psi +
\Psi^*\hat{H}\left(\frac{i}{\hbar}\delta\ph \op{l}_z\right)\Psi
- \Psi^*\left(\frac{i}{\hbar}\delta\ph \op{l}_z\right)\hat{H}\Psi \\
&+\underbrace{\frac{1}{\hbar^2}\delta\ph^2\Psi^*\op{l}_z\hat{H}\op{l}_z\Psi}_{ 
= 0, \text{ da quadratisch in $\delta\ph$}.}\\
&= \int\dV \Psi^*\left[\hat{H} - \frac{i}{\hbar}\left(\op{l}_z\hat{H} -
\hat{H}\op{l}_z\right)\right]\Psi\\ &\overset{!}{=} \lin{\hat{H}},
\end{align*}
dann gilt die Gleichheit genau dann, wenn $\hat{l}_z$ mit dem Hamilton
vertauscht,
\begin{align*}
\nrm{\hat{l}_z,\hat{H}} = 0.
\end{align*}
Rotationssymmetrie um die $z$-Achse impliziert also wie in der klassischen
Mechanik die Erhaltung des Drehimpulses um die $z$-Achse.\bsphere
\end{bspn}

\subsection{Relativistische Invarianz}

Eine \emph{Lorenztransformation}\footnote{Hendrik Antoon Lorentz (* 18. Juli
1853 in Arnheim; † 4. Februar 1928 in Haarlem) war ein niederländischer
Mathematiker und Physiker und Nobelpreisträger für Physik des Jahres 1902.}, 
erhält den Abstand zweier Punkte in der Raumzeit. Eine Größe, die sich unter
einer Lorentztransformation nicht ändert, heißt \emph{lorentzinvariant}.

Die Lorenztransformation ist somit eine Symmetrieoperation $\hat{S}$ mit einer
Invarianten $s$, dem raumzeitlichen Abstand, gegeben durch,
\begin{align*}
s^2 = c^2(t_1-t_2)^2 - (x_1-x_2)^2 - (y_1-y_2)^2 - (z_1-z_2)^2 = \const.
\end{align*}
\begin{bspn}
Lineare Bewegung in $x$-Richtung mit Geschwindigkeit $v$,
\begin{align*}
\hat{S}:\; & ct' = c(\cosh \alpha)t - x(\sinh\alpha)\\
& x' = x\cosh \alpha - c(\sinh \alpha)t\\
& y' = y\\
& z' = z
\end{align*}
wobei
\begin{align*}
&\cosh \alpha = \frac{1}{\sqrt{1-\beta^2}},\quad
\sin\alpha = \frac{\beta}{\sqrt{1-\beta^2}},\quad
 \beta=\frac{v}{c}.
\end{align*}
Da $\sinh^2 \alpha - \cosh^2 \alpha = 1$ gilt $\hat{S}(s^2)=s^2$.\bsphere
\end{bspn}

Um in der Raumzeit geschickt rechnen zu können, führen wir verallgemeinerte
Vektoren ein, die \emph{Vierervektoren},
\begin{align*}
\vec{x}^\mu = \left\{ct,x,y,z \right\}.
\end{align*}
Für diese Vektoren definieren wir ein neues Skalarprodukt
\begin{align*}
\abs{\vec{x}}^2 = \lin{\vec{x}_\mu,\vec{x}^\mu} = \sum\limits_{\mu=0}^3
x_\mu x^\mu = c^2t^2 - x^2-y^2-z^2.
\end{align*}
Die Energie $E=\frac{mc^2}{\sqrt{1-\beta^2}}$ und der Impuls $\vec{p} =
\frac{m\vec{v}}{\sqrt{1-\beta^2}}$ bilden den Viererimpuls,
\begin{align*}
\vec{p}^\mu = \left\{ \frac{E}{c}, p_x, p_y, p_z\right\}.
\end{align*}
Durch obige Definition ist das Skalarprodukt invariant unter
Lorentztransformationen. Insbesondere gilt
\begin{align*}
\lin{\hat{S}p_\mu,\hat{S}p^\mu} = \lin{p_\mu,p^\mu} = \sum\limits_{\mu=0}^3
p_\mu p^\mu = \frac{E^2}{c^2} - p^2 = mc^2 = \const.\tag{*}
\end{align*}

Insbesondere erhalten wir für $p=0$ die Ruheenergie $E_0 = mc^2$.
Durch Lorentztransformationen kann somit Ruhemasse in kinetische Energie
umgewandelt werden und umgekehrt.
\begin{bspn}
Bei einem Stoßprozess sind $E_\tot$ und $\vec{p}_\tot$ erhalten, aber $m_\tot =
\sum_i m_i\neq \const$. Insbesondere kann kinetische Energie in Ruhemasse und
Ruhemasse in kinetische Energie umgewandelt werden.
\begin{figure}[!htpb]
  \centering
\begin{pspicture}(-0.1,-2)(3,2)
\psline{->}(0,1.5)(2,1.5)
\psline[linestyle=dotted]{->}(0,0)(2,0)
\psline{->}(0,-1.5)(2,-1.5)

\rput(2.5,1.5){\color{gdarkgray}$E_+$}
\rput(2.5,-1.5){\color{gdarkgray}$E_-$}
\rput(2.5,0){\color{gdarkgray}$0$}
\end{pspicture} 
  \caption{Energie-Masse-Äquivalenz.}
\end{figure}
\textit{Zahlenbeispiel:} $10^{-4}g\entspr 2500 \mathrm{kWh}$.

Aus (*) folgt,
\begin{align*}
E = \pm c\sqrt{p^2+m^2c^2}.
\end{align*}
Dass zwei Vorzeichen $\pm$ möglich sind, kündigt die Existenz von Antiteilchen
an. Beispielsweise kann für $E_\kin > 2m_ec^2$ bei einem Stoßprozess ein
$e^+,e^-$-Paar erzeugt werden.\bsphere
\end{bspn}

\subsection{Der Spin}
Zur Charakterisierung eines Teilchens benötigt man folgende Größen:

%\begin{table}
\begin{tabular}[H]{ll}
$m_\text{träge}$, $m_\text{schwer}$ & Gravitation\\
$Q$ & Elektromagnetismus\\
Farbladungen & Starke + Schwache WW\\
$\vec{p},\vec{L},E$ & Beschreiben Bewegung des Schwerpunkts\\
Spin & innerer Freiheitsgrad.
\end{tabular}
%\end{table}

Wir wollen nun der Frage nachgehen wie viele ``Sorten'' Spin es geben kann.
Klassisch wird der Spin eines Objekts oft mit dem Eigendrehimpuls
identifiziert. Bei Elementarteilchen mit
verschwindender Ausdehnung ($r\to0$) versagt jedoch die Vorstellung. Außerdem
würde eine klassisch rotierende Ladung ein anderes magnetisches Moment erzeugen
als das experimentell beobachtete (siehe $g$-Faktor des Elektonsr
$g=2.001\ldots$).

In der Quantenmechanik definiert man daher \emph{Spinoperatoren}, die den
selben Vertauschungsrelationen wie Drehimpulsoperatoren folgen.
\begin{align*}
&\nrm{\hat{s}_i,\hat{s}_j} = i\hbar\ep_{ijk} \hat{s}_k,\qquad
\ep_{123} =
\begin{cases}
1, & \text{zyklische Vertauschung von (1,2,3)},\\
-1, & \text{antizyklische Vertauschung},\\
0, & \text{sonst}.
\end{cases}
\end{align*}
\begin{bspn} Im dreidimensionalen Fall erhalten wir,
\begin{align*}
\nrm{\hat{s}_x,\hat{s}_y} = i\hbar \ep_{xyz}\hat{s}_z.\bsphere
\end{align*}
\end{bspn}
Das skalare Quadrat des Spins ist gegeben durch,
\begin{align*}
\hat{s}^2 = \underbrace{\hat{s}_x^2 + \hat{s}_y^2 + \hat{s}_z^2}_{\ge 0}.\tag{*}
\end{align*}
Die Eigenwerte von $\hat{s}$ werden mit $s$ die von $\hat{s}_z$ mit
$m_s$ bezeichnet. Man kann zeigen dass die gemeinsamen Eigenzustände
$\ket{s,m_s}$ von $\hat{s}$ und $\hat{s}_z$ ein VONS bilden. Da die Eigenwerte
von $\hat{s}^2$ positiv sind (siehe (*)), lassen sie sich als $s(s+1)$
schreiben, womit wir folgende Eigenwertgleichungen erhalten,
\begin{align*}
&\hat{s}^2\ket{s,m_s} = s(s+1)\hbar^2\ket{s,m_s},\\
&\hat{s}_z\ket{s,m_s} = m_s\hbar\ket{s,m_s}.
\end{align*}
\begin{bemn}
Da $\nrm{\hat{s}^2,\hat{s}_z}=0$, sind $\hat{s}^2$ und $\hat{s}_z$ zwei
gleichzeitig beobachtbare Erhaltungsgrößen.\maphere
\end{bemn}
Wir wollen nun mit Hilfe der Eigenwertgleichungen $s$ und $m_s$ konkret
berechnen. Dazu halten wir fest,
\begin{align*}
&\hat{s}^2\ket{s,m_s} = f(s,m_s)\ket{s,m_s},\\
&\hat{s}^2-\hat{s}_z^2 = \underbrace{\hat{s}_x^2 + \hat{s}_y^2}_{\ge 0}.
\end{align*}
Es gilt also $f(s,m_s)\ge \hbar^2 m_s^2$, d.h. $m_s$ hat einen Maximalwert.

Oft ist es geschickt anstatt mit $\hat{s}_x$, $\hat{s}_y$ oder $\hat{s}_z$ mit
einer der folgenden Linearkombinationen zu rechnen.
\begin{defnn}
\emph{Leiteroperatoren}
\begin{align*}
&\hat{s}_+ := \hat{s}_x + i\hat{s}_y,&& \text{``Aufsteigeoperator''},\\
&\hat{s}_- := \hat{s}_x - i\hat{s}_y,&& \text{``Absteigeoperator''}.
\end{align*}
Als Linearkombinationen von $\hat{s}_x$ und $\hat{s}_y$ erfüllen sie
die Vertauschungsrelationen,
\begin{align*}
&\nrm{\hat{s}^2,\hat{s}_\pm} = 0,&&\text{d.h. }\hat{s}_\pm\text{ ändert $s$
nicht}.\\ &\nrm{\hat{s}_z,\hat{s}_\pm} =\pm\hbar\hat{s}_\pm, &&\text{d.h.
$\hat{s}_\pm$ ändert $m_s$}.\fishhere
\end{align*}
\end{defnn}
Betrachten wir nun die zwei Zustände $\hat{s}_\pm\ket{s,m_s} $ und die Wirkung
von $\hat{s}_z$ und $\hat{s}^2$,
\begin{align*}
\hat{s}_z\hat{s}_\pm\ket{s,m_s} &= \left(\hat{s}_\pm\hat{s}_z +
\nrm{\hat{s}_z,\hat{s}_\pm}\right)\ket{s,m_s}
= \left(\hat{s}_\pm\hat{s}_z \pm
\hbar\hat{s}_\pm\right)\ket{s,m_s}
\\ &= \left( \hbar m_s\hat{s}_\pm \pm
\hbar\hat{s}_\pm\right)\ket{s,m_s}
= \hbar (m_s \pm 1)\hat{s}_\pm\ket{s,m_s},\\
\hat{s}^2\hat{s}_\pm\ket{s,m_s} &= \hat{s}_\pm\hat{s}^2\ket{s,m_s} =
\hbar^2 l(l+1)\hat{s}_\pm\ket{s,m_s}.
\end{align*}
Da $m_s$ einen Maximalwert besitzt, erhalten wir so:

$\hat{s}_+\ket{s,m_s}$ ist Eigenzustand zu $\hat{s}^2$ und $\hat{s}_z$ mit
den Eigenwerten $\hbar^2s(s+1)$ bzw. $\hbar (m_s+1)$ oder der Nullvektor.

$\hat{s}_-\ket{s,m_s}$ ist Eigenzustand zu $\hat{s}^2$ und $\hat{s}_z$ mit
den Eigenwerten $\hbar^2s(s+1)$ bzw. $\hbar (m_s-1)$ oder der Nullvektor.

Betrachten wir die Norm 
$\norm{\hat{s}_\pm\ket{s,m_s}}^2=\bra{s,m_s}\hat{s}_\mp\hat{s}_\pm\ket{s,m_s}$
und verwenden,
\begin{align*}
\hat{s}_\mp\hat{s}_\pm = (\hat{s}_x\mp i\hat{s}_y)(\hat{s}_x\pm i\hat{s}_y) =
\hat{s}_x^2+\hat{s}_y^2 \pm i\nrm{\hat{s}_x,\hat{s}_y} = \hat{s}^2 -
\hat{s}_z^2 \mp \hbar\hat{s}_z
\end{align*}
erhalten wir,
\begin{align*}
\norm{\hat{s}_\pm\ket{s,m_s}}^2 = \left[l(l+1) - m_s(m_s\pm1)\right]\hbar^2 \ge
0.\tag{**}
\end{align*}
D.h. $-s\le m_s \le s$.

Aufgrund der Einschränkung der möglichen $m_s$ muss $\hat{s}_+\ket{s,s}$ der
Nullvektor sein, denn $\hat{s}_+$ erhöht $m_s$. Daher ist
$\norm{\hat{s}_+\ket{s,s}}^2=0 \Rightarrow m_{s,\text{max}} = s$.

Analog folgt $m_{s,\text{min}} = -s$. Alle möglichen $m_s$ haben daher einen
ganzzahligen Abstand zu $l$, denn die Leiteroperatoren erhöhren bzw.
erniedrigen stets um ganze Zahlen und müssen die Gleichung (**) erfüllen.

Daher muss $-s = m_{s,\text{min}}$ ganzzahligen Abstand zu $s$ haben. D.h. $2s$
ist ganzzahlig und daher sind für $s$ nur halbe oder ganze Zahlen zugelassen. 

\begin{figure}[H]
\centering
\psset{unit=0.6cm}
\begin{minipage}{0.4\linewidth}
\begin{pspicture}(-1.6,-3.5)(4.8,3.8)
\psaxes[labels=none,ticks=none,ticksize=1pt]{->}%
 (0,0)(-0.5,-3.5)(4.5,3.5)[\color{gdarkgray}$s$,-90][\color{gdarkgray}$m_s$,0]
 
\psdots[linecolor=darkblue](0,0)(2,0)(1,1)(1,-1)(2,2)(2,-2)(3,1)(3,3)(3,-1)(3,-3)

\psyTick(3){\color{gdarkgray}3/2}
\psyTick(2){\tiny{\color{gdarkgray}1}}
\psyTick(1){\small\color{gdarkgray}1/2}
\psyTick(-1){\small\color{gdarkgray}-1/2}
\psyTick(-2){\small\color{gdarkgray}-1}
\psyTick(-3){\small\color{gdarkgray}-3/2}
\psxTick(1){\small\color{gdarkgray}1/2}
\psxTick(2){\small\color{gdarkgray}1}
\psxTick(3){\small\color{gdarkgray}3/2}

\end{pspicture}
\end{minipage}
\begin{minipage}{0.4\linewidth}
\begin{tabular}{l|cccc}
$s$ & $0$ & $\frac{1}{2}$ & $1$ & \ldots\\\hline
$m_s$ & $0$ & $-\frac{1}{2},\; \frac{1}{2}$ & $-1,0,1$ & \ldots\\
\end{tabular}
\end{minipage}
\caption{Mögliche Wertepaare für $(s,m_s)$.}
\end{figure}

Teilchen mit geradzahligem Spin $s=0,1,2,\ldots$ werden als \emph{Bosonen}
bezeichnet, Teilchen mit halbzahligem Spin
$s=\frac{1}{2},\frac{3}{2},\ldots$ als \emph{Fermionen}.

\begin{bemn}
Bei dieser Überlegung haben wir nicht über die Interpretation von $m_s$ und $s$
gesprochen, sondern nur ausgehend von der Isotropie  in 3 Dimensionen mögliche
``Arten'' von Punktteilchen diskutiert, die diese Isotropie nicht verletzen. Da
die Drehimpulsoperatoren Erhaltungsgrößen beschreiben, haben wir  ihre Struktur
als ``Vorbild'' für die Spinoperatoren hergenommen.

Ein Punktteilchen mit Spin $s$ kommt also im Prinzip in $2s+1$ ``Arten'' vor.
Teilchen, welche in einer (un)geraden Anzahl von ``Arten'' vorkommen, nennt man
\emph{Fermionen} (\emph{Bosonen}). In zwei Dimensionen würden die Überlegungen
übrigens zu einem anderen Schluss führen, denn dort gibt es ein Kontinuum von
Teilchen (Anyonen). Die Spinoperatoren in 2 Dimensionen sind trivial.\maphere 
\end{bemn}

\subsection{Permutationssymmetrie von Bosonen und Fermionen}

Aus der Sicht der klassischen Mechanik lassen sich alle Teilchen zu jeder Zeit
genau unterscheiden.

\begin{figure}[!ht]
  \centering
\begin{pspicture}(-0.1,-2)(4.34,2)
\pscircle(0.25,1.68){0.3}
\pscircle(4,1.56){0.3}
\psline{->}(0.46,1.35)(1.68,-0.19)
\psline{->}(3.76,1.15)(1.9,-0.21)
\psdots[linecolor=darkblue,dotsize=0.2](1.78,-0.33)
\psline[linestyle=dotted]{->}(1.6,-0.55)(0.52,-1.97)
\psline[linestyle=dotted]{->}(1.98,-0.53)(3.62,-1.81)
\rput(0.25,1.68){\color{gdarkgray}$1$}
\rput(4,1.56){\color{gdarkgray}$2$}
\end{pspicture} 
  \caption{Stoßprozess in der klassischen Mechanik.}
\end{figure}
Da man den Ort beliebig genau auflösen kann, lässt sich determinieren, ob
Teilchen 1 oder Teilchen 2 nach rechts abgelenkt wird.

In der Quantenmechanik sind Teilchen 1 und Teilchen 2 bei gleicher Konfiguration
jedoch ununterscheidbar, d.h. eine Vertauschung ändert die Physik nicht, weshalb
die Streuung im klassischen Sinn einen nicht determinierbaren Ausgang hat, es
kommt zu Überlagerungszuständen.

\subsubsection{Mathematische Formulierung}

Wir wollen nun die Vertauschung von Objekten mathematisch formulieren.

\begin{figure}[!ht]
  \centering
\begin{pspicture}(0,-1.12)(8.8,1.5)
\psline(0.0,-0.9)(3.7,-0.9)
\psline(1.66,-1.1)(1.66,1.1)

\psframe[linecolor=darkblue](3.42,0.52)(3.04,0.14)
\pscircle[linecolor=yellow](0.56,0.6){0.22}
\psline{->}(1.66,-0.9)(3.02,0.1)
\psline{->}(1.66,-0.9)(0.76,0.36)

\psline(5.0,-0.9)(8.7,-0.9)
\psline(6.66,-1.1)(6.66,1.1)

\psframe[linecolor=yellow](5.7409377,0.78)(5.3409376,0.38)
\pscircle[linecolor=darkblue](8.200938,0.28){0.22}

\psline{->}(6.66,-0.9)(8.02,0.1)
\psline{->}(6.66,-0.9)(5.76,0.36)

\psbezier[linecolor=darkblue]{->}(3.2058887,0.7)(3.7058887,1.4)(4.685889,1.4)(5.405889,0.85)

\rput(3.6,0){\color{gdarkgray}$\vec{r}_1$}
\rput(0.3,0.25){\color{gdarkgray}$\vec{r}_2$}

\rput(8.6,0){\color{gdarkgray}$\vec{r}_1$}
\rput(5.3,0.2){\color{gdarkgray}$\vec{r}_2$}

\rput(4.3,1){\color{gdarkgray}$\hat{P}$}
\end{pspicture}
  \caption{Vertauschungsoperation bei zwei Objekten.}
\end{figure}

Dabei ist zu berücksichtigen, dass eine zweifache Vertauschung wieder zum
Ausgangszustand zurückführt. Die Eigenwertgleichung des Vertauschungsoperators
hat daher die Gestalt,
\begin{align*}
&\hat{P}(\Psi(\vec{r}_1,\vec{r}_2)) = p\Psi(\vec{r}_2,\vec{r}_1).\\
&\hat{P}^2(\Psi(\vec{r}_1,\vec{r}_2)) = p\hat{P}(\Psi(\vec{r}_2,\vec{r}_1))
= p^2\Psi(\vec{r}_1,\vec{r}_2) = \Psi(\vec{r}_1,\vec{r}_2).
\end{align*}
Es muss gelten $p=\pm1$ weshalb wir zwei Arten von Parität erhalten,
\begin{align*}
&\hat{P}\Psi = +\Psi,\qquad \text{symmetrisch},\\
&\hat{P}\Psi = -\Psi,\qquad \text{antisymmetrisch}.
\end{align*}

\begin{figure}[!ht]
  \centering
\begin{pspicture}(0,-1.59)(4.92,1.59)
\psellipse[linecolor=darkblue](1.36,0.04)(1.36,0.71)
\psellipse[linecolor=yellow](3.56,0.02)(1.36,0.71)

\rput(2.44,0.035){\color{gdarkgray}+}
\rput(4.66,0.055){\color{gdarkgray}-}
\rput(0.26,0.055){\color{gdarkgray}-}

\rput(3.55,-0.9){\color{gdarkgray}2}
\rput(1.33,0.955){\color{gdarkgray}1}

\psdots[dotsize=0.2,linecolor=darkblue](1.36,0.03)
\psdots[dotsize=0.2,linecolor=yellow](3.56,0.03)

\psarc{<-}(2.66,-0.08){1.42}{213.2317}{306.25385}
\psarc{<-}(2.32,0.15){1.42}{30.650667}{129.61069}
\end{pspicture}
\qquad
\begin{pspicture}(0,-1.59)(4.92,1.59)
\psellipse[linecolor=yellow](1.36,0.04)(1.36,0.71)
\psellipse[linecolor=darkblue](3.56,0.02)(1.36,0.71)

\rput(2.44,0.035){\color{gdarkgray}-}
\rput(4.66,0.055){\color{gdarkgray}+}
\rput(0.26,0.055){\color{gdarkgray}+}

\rput(3.55,-0.9){\color{gdarkgray}1}
\rput(1.33,0.955){\color{gdarkgray}2}

\psdots[dotsize=0.2,linecolor=yellow](1.36,0.03)
\psdots[dotsize=0.2,linecolor=darkblue](3.56,0.03)

%\psarc{<-}(2.66,-0.08){1.42}{213.2317}{306.25385}
%\psarc{<-}(2.32,0.15){1.42}{30.650667}{129.61069}
\end{pspicture} 
  \caption{Zur Permutationssymmetrie.}
\end{figure}

\begin{bspn}
Im Fall von zwei Teilchen ist die Eigenfunktion des Vertauschungsoperators
gegeben durch,
\begin{align*}
\Psi_\pm = \frac{1}{\sqrt{2}}\left(\Psi_1(\vec{r}_1)\Psi_2(\vec{r}_2) \pm
\Psi_1(\vec{r}_2)\Psi_2(\vec{r}_1) \right),
\end{align*}
wobei bei Bosonen ein $+$ und bei Fermionen ein $-$ auftritt.\bsphere
\end{bspn}

\begin{bemn}
Von Teilchen 1 aus gesehen entspricht die Vertauschung mit Teilchen 2 gerade
einer Rotation um $\pi$. Wie wir auf  S. \pageref{subsubsec:Azimutalgleichung}
sehen werden, sind die $m_s$ Eigenfunktionen von der Form,
\begin{align*}
\Psi \sim e^{im_s\ph}.
\end{align*}
Für $m_s = \frac{1}{2}$ bedeutet das für ein Teilchen, dass man $\ph=4\pi$
(d.h. zwei Umdrehungen) benötigt, um zum gleichen Zustand zurückzukommen. Für
zwei Teilchen, die jeweils um $\pi$ umeinander rotiert werden
(Vertauschung), bedeutet das, dass ihre relative Phase
$\exp\left(i\frac{1}{2}(\pi_1+\pi_2)\right) = -1$ beträgt.\maphere
\end{bemn}

\subsubsection{Pauli-Verbot}
Betrachtet man zwei Fermionen, die durch die Wellenfunktionen
\begin{align*}
\Psi_1(\vec{r}_1),\; \Psi_2(\vec{r}_2),
\end{align*}
beschrieben werden, am gleichen Ort 
$\vec{r}_1=\vec{r}_2=\vec{r}$, so gilt aufgrund der Parität für die
Zweiteilchenwellenfunktion
\begin{align*}
\Psi_-(\vec{r}) = \frac{1}{\sqrt{2}}\left(\Psi_1(\vec{r})\Psi_2(\vec{r})-
\Psi_2(\vec{r})\Psi_1(\vec{r})\right) = 0.
\end{align*} 
Die Wahrscheinlichkeit bei einer Messung beide Teilchen am gleichen Ort
vorzufinden ist also Null!


\begin{figure}[!htbp]
\centering
\begin{tabular}{l|c|c}
&\color{darkblue}\textbf{Fermionen} & \color{darkblue}\textbf{Bosonen}\\\hline
\textbf{Parität} &
$\Psi(\vec{r}_1,\vec{r}_2) = -\Psi(\vec{r}_2,\vec{r}_1)$ &
$\Psi(\vec{r}_1,\vec{r}_2) = \Psi(\vec{r}_2,\vec{r}_1)$\\
& ungerade & gerade \\\hline
\textbf{Spin} &
$s=\frac{1}{2},\frac{3}{2},\frac{5}{2},\ldots$
&
$s=0,1,2,\ldots$\\
& halbzahlig &
ganzzahlig\\\hline
&
$
\Psi(\vec{r}_1,\vec{r}_1) = -\Psi(\vec{r}_1,\vec{r}_1)
\Rightarrow \Psi(\vec{r}_1,\vec{r}_1) \equiv 0$
 & $\Psi(\vec{r}_1,\vec{r}_1) \neq 0$\\
& Pauli Prinzip
& (Durchdringen erlaubt)\\\hline &
$e^-$, $p^+$, $n^0$, $\gamma_e$, $\mu$&
Photon $\gamma$, Schwache WW $W^\pm$, $Z$\\
& $D$, ${}^3He$, ${}^6 Li$, ${}^{53} Cr$ &
 $H$, ${}^4 He$, ${}^{87}Rb$, ${}^{52} Cr$, $C_{60}$, $O_2$
\end{tabular}
\caption{Übersicht Fermionen, Bosonen.}
\end{figure}

\subsection{Parität}
Mit Parität wird eine Symmetrieeigenschaft eines Systems unter Punktspiegelung
bezeichnet. Der Paritätsoperator ist gegeben durch
\begin{align*}
\hat{P}: \vec{r} \mapsto -\vec{r}.
\end{align*}

\begin{figure}[!ht]
  \centering
\begin{pspicture}(0,-1.5092187)(5.44,1.5092187)
\psline{->}(1.16,-0.28921875)(1.16,0.81078124)
\psline{->}(1.14,-0.26921874)(2.32,-0.26921874)
\psline{->}(1.16,-0.26921874)(0.42,-0.7892187)
\psline[linestyle=dotted,dotsep=0.06cm](1.16,-0.24921875)(1.16,-1.3492187)
\psline[linestyle=dotted,dotsep=0.06cm](1.18,-0.26921874)(0.0,-0.26921874)
\psline[linestyle=dotted,dotsep=0.06cm](1.16,-0.26921874)(1.9,0.25078124)
\psline{->}(4.26,-0.24921875)(4.26,-1.3492187)
\psline{->}(4.28,-0.26921874)(3.1,-0.26921874)
\psline{->}(4.26,-0.26921874)(5.0,0.25078124)
\psline[linestyle=dotted,dotsep=0.06cm](4.26,-0.28921875)(4.26,0.81078124)
\psline[linestyle=dotted,dotsep=0.06cm](4.24,-0.26921874)(5.42,-0.26921874)
\psline[linestyle=dotted,dotsep=0.06cm](4.26,-0.26921874)(3.52,-0.7892187)

\psdots(1.16,-0.26921874)
\psdots(4.26,-0.26921874)

\psbezier[linecolor=darkblue]{->}(1.94,0.73078126)(2.34,1.2307812)(3.0,1.2307812)(3.48,0.67078125)

\rput(2.2878125,-0.43921876){\color{gdarkgray}$y$}
\rput(0.52546877,-0.89921874){\color{gdarkgray}$x$}
\rput(0.9934375,0.86078125){\color{gdarkgray}$z$}
\rput(4.44125,-1.3592187){\color{gdarkgray}$z'$}
\rput(5.1625,0.26078126){\color{gdarkgray}$x'$}
\rput(3.19375,0){\color{gdarkgray}$y'$}
\rput(2.7307813,1.3407812){\color{gdarkgray}$\hat{P}$}

\end{pspicture} 
  \caption{Paritätsspiegelung eines rechtshändigen Koordinatensystem in ein
  linkshändiges.}
\end{figure}
Zweifache Anwendung der Symmetrieoperation führt das System in sich selbst über,
\begin{align*}
\hat{P}^2 = \Id,\qquad\Rightarrow \text{Eigenwerte }\pm1.
\end{align*}
Polare Vektoren ändern unter der Paritätsoperation ihr Vorzeichen,
\begin{align*}
&\hat{P}\vec{r} = -\vec{r},\\
&\hat{P}\vec{p} = -\vec{p},
\end{align*}
während das Vorzeichen axialer Vektoren invariant ist,
\begin{align*}
&\hat{P}\vec{L} = \hat{P}(\vec{r}\times\vec{p}) = -\vec{r}\times -\vec{p} =
\vec{L}.
\end{align*}
Magnetische und elektrische Felder und Dipole transformieren wie folgt,
\begin{align*}
&\hat{P}\vec{E} = -\vec{E},
&& \op{d} = e\op{r} \mapsto -\op{d}\\
&\hat{P}\vec{B} = \vec{B},
&& \op{\mu} = e\left(\op{r}\times\op{p}\right) \mapsto \op{\mu}
\end{align*}

Falls $\nrm{\hat{H},\hat{P}}=0$, so ist $\hat{P}$ eine Symmetrieoperation und
die Parität erhalten, d.h. die Physik im Spiegel betrachtet ist nicht von der
Realtität unterscheidbar.

\begin{bspn}
\begin{enumerate}[label=\arabic{*}.)]
  \item Die elektromagentische Wechselwirkung erhält die Parität. Z.B.
\begin{align*}
&\op{H} = \ldots\hat{B}^2 + \ldots\hat{E}^2\\
&\op{H} \sim \vec{d}\vec{E},\quad \vec{H} \sim \vec{\mu}\vec{B}.
\end{align*}
  \item Für die schwache Wechselwirkung ist $\nrm{\hat{H},\hat{P}}\neq0$, d.h.
  die Parität ist keine universelle Symmetrie.
  \item Im Wasserstoff-Atom ist die elektromagnetische Wechselwirkung dominant,
  daher sind die Wasserstoff-Zustände Eigenfunktionen zu $\op{P}$.

\begin{figure}[H]
\centering
\begin{pspicture}(-0.5,-2.2)(10,2.2)

\pscircle(1.2,0.2){1.18}
\psdots[dotsize=0.12](1.2,0.2)

\rput(1.2,2){\color{gdarkgray}$l=0$}

\rput(1.2,-1.4){\color{gdarkgray}$P=(-1)^l=+1$}

\rput(1.2,-1.8){\color{gdarkgray}symmetrisch unter $\hat{P}$}

\psbezier(6.1,1.85)(4.7,1.85)(7.5,-1.2528125)(6.1,-1.27)(4.7,-1.2928125)(7.54,1.85)(6.1,1.85)
\psbezier(8.84,1.85)(7.42,1.85)(10.22,-1.2528125)(8.85,-1.27)(7.42,-1.2928125)(10.26,1.85)(8.84,1.85)

\psline{->}(6.52,0.28)(8.4,0.28)

\rput(6.1,1.45){+}
\rput(6.1,-0.7){-}

\rput(8.85,1.45){-}
\rput(8.85,-0.7){+}

\rput(7.43,2){\color{gdarkgray}$l=1$}

\rput(7.43,0.6571875){\color{gdarkgray}$\hat{P}$}
\rput(7.43,-1.4){\color{gdarkgray}$P=(-1)^l=-1$}

\rput(7.43,-1.8){\color{gdarkgray}assymmetrisch unter $\hat{P}$}
\end{pspicture} 
  \caption{Symmetrie der Zustände eines Wasserstoffatoms.\bsphere}
\end{figure}
\end{enumerate}
\end{bspn}

\subsubsection{Paritätserhaltung bei Übergängen von Zuständen}

Die Wahrscheinlichkeit für den Übergang vom Zustand $i$ in den
Zustand $f$ unter $\hat{O}$ ist gegeben durch das Matrixelemnt,
\begin{align*}
A_{i\to f} = \int\dV \Psi_f \hat{O}\Psi_i.
\end{align*}
Ob ein Übergang möglich ist, kann man an der Symmetrie der
Wellenfunktion erkennen. Ist $\hat{P}\Psi_f \hat{O}\Psi_i=-1$, so ist 
die Wellenfunktion ungerade und das Integral verschwindet, der Übergang ist also
verboten!

Betrachten wir nun die Eigenwerte $P_f$, $P_O$ und $P_i$ von $\hat{P}$
bezüglich $\Psi_f$, $\hat{O}$ und $\Psi_i$, so ergibt sich als Auswahlkriterium
\begin{align*}
\begin{cases}
P_fP_OP_i = 1,&\text{Übergang möglich},\\
P_fP_OP_i = -1,&\text{Übergang unmöglich}.
\end{cases}
\end{align*}

\begin{pspicture}(0,-1.6471875)(6.28,1.7)
\psline{->}(0.9,-1.2228125)(0.9,0.7771875)
\psline{->}(0.0,-0.3028125)(2.0,-0.3028125)
\psline{->}(3.04,-1.2228125)(3.04,0.7771875)
\psline{->}(2.14,-0.3028125)(4.14,-0.3028125)
\psline{->}(5.16,-1.2228125)(5.16,0.7771875)
\psline{->}(4.26,-0.3028125)(6.26,-0.3028125)


\psbezier[linecolor=darkblue](0.04,-0.28388345)(0.04,-0.28388345)(0.096880935,0.19622183)(0.46,0.1971875)(0.82311904,0.19815317)(1.02,-0.8028125)(1.36,-0.8028125)(1.7,-0.8028125)(1.8,-0.30315688)(1.8,-0.30315688)
\psline[linecolor=darkblue](2.24,-1.0228125)(3.96,0.5171875)
\psbezier[linecolor=darkblue](4.28,-0.2828125)(4.727921,-0.2828125)(4.8657427,0.2371875)(5.1586137,0.2371875)(5.451485,0.2371875)(5.589307,-0.3028125)(6.0,-0.2828125)


\rput(2.65,1.4671875){\color{gdarkgray}$\int\dV \Psi_f \hat{O}\Psi_i$}

\psline{->}(2.7,1.2571875)(1.62,0.8371875)
\psline{->}(3.04,1.2371875)(3.04,0.7971875)
\psline{->}(3.38,1.2571875)(4.46,0.8371875)

\rput(0.895,-1.4928125){\color{gdarkgray}$P_f=-1$}
\rput(3.065,-1.4928125){\color{gdarkgray}$P_0=-1$}
\rput(5.155,-1.4928125){\color{gdarkgray}$P_i=1$}
\end{pspicture} 

Für das magnetische und das elektrische Dipolmoment gilt,
\begin{align*}
&\vec{d}= e\vec{r}\mapsto -\vec{d},&&\Rightarrow \vec{H}=\vec{d}\cdot\vec{E}
= \vec{d}\cdot\vec{E}\\
&\vec{\mu}= e\left(\vec{r}\times\vec{p}\right)\mapsto\vec{\mu}, &&\Rightarrow
\vec{H}=\vec{\mu}\cdot\vec{B} =
\vec{\mu}\cdot\vec{B}.
\end{align*}
Wir erhalten somit die Auswahlregeln
\begin{align*}
&\lin{f|\vec{d}|i}\neq 0\Leftrightarrow P_i\neq P_f\tag{E1}\\
&\lin{f|\vec{\mu}|i}\neq 0\Leftrightarrow P_i= P_f\tag{M1}
\end{align*}

\begin{figure}[!ht]
\centering
\begin{pspicture}(-0.1,-1.27)(2.9,1.27)
\psline(0.3925,1.17)(0.3925,-1.25)
\psline(0.3925,-1.23)(2.0325,-1.23)
\psline(2.0125,-1.23)(2.0125,1.19)
\psline(0.3925,-0.93)(2.0125,-0.93)
\psline(0.3925,-0.43)(2.0125,-0.43)
\psline(0.3925,0.77)(2.0125,0.77)

\psbezier[linecolor=darkblue](0.3925,-0.91)(0.6725,-0.63)(1.7325,-0.71)(2.0125,-0.93)
\psbezier[linecolor=darkblue](0.3925,-0.43)(0.5925,-0.13)(0.9125,-0.15)(1.2125,-0.43)(1.5125,-0.71)(1.7125,-0.71)(1.9925,-0.43)
\psbezier[linecolor=darkblue](0.3925,0.77)(0.6125,1.25)(0.73669755,1.21)(0.93916667,0.77)(1.1416358,0.33)(1.2925,0.33)(1.5060803,0.77)(1.7196605,1.21)(1.8125,1.25)(2.0125,0.75)

\psline[linestyle=dotted,dotsep=0.06cm]{->}(0.5725,-0.43)(0.5725,-0.93)
\psline[linestyle=dotted,dotsep=0.06cm]{->}(1.7725,0.77)(1.7725,-0.41)

\rput(2.4639063,0.84){\small\color{gdarkgray}$n=2$}

\rput(2.449375,-0.38){\small\color{gdarkgray}$n=1$}

\rput(2.465,-0.9){\small\color{gdarkgray}$n=0$}

\rput(1.4867188,0.26){\small\color{gdarkgray}$E1$}
\rput(0.16671875,-0.66){\small\color{gdarkgray}$E1$}
\end{pspicture}
\qquad
\begin{pspicture}(0,-0.6)(3.1,0.6)
\psframe(3,0.6)(0.0,-0.6)
\rput[l](0.2,0.31){$E1:$}
\rput[l](0.2,-0.31){$M1:$}
\rput[l](1.15,0.31){$\Delta n=1,3,...$}
\rput[l](1.15,-0.31){$\Delta n=2,4,...$}
\end{pspicture}
\caption{Mögliche Übergange mit $\Delta n$ ungerade.}
\end{figure}


\begin{bspn}
Betrachten wir den Übergang von ${}^3P_2$ zu ${}^3S_1$ im Wasserstoffatom.

\begin{figure}[!ht]
  \centering
  \begin{pspicture}(-2.1,-1.2)(7.4,1.2)

\psbezier{->}(2.9721875,0.7107813)(2.4521875,-0.08921875)(2.4321876,1.0707812)(1.8121876,0.41078126)
\psbezier{<-}(1.9121875,0.01078125)(2.5521874,-0.44921875)(1.8121876,-0.68921876)(2.4121876,-0.92921877)
\psbezier{->}(0.3121875,0.45078126)(0.8321875,-0.12921876)(0.8521875,1.0307813)(1.3321875,0.45078126)

\rput(1.6289062,0.20578125){\color{darkblue}\Large ${}^3S_1$}

\rput[l](3.1470313,0.9607813){\color{gdarkgray}$S\entspr l=0$ Bahndrehimpuls}
\rput[l](2.6,-0.9792187){\color{gdarkgray}$\abs{\vec{j}}=\abs{\vec{s}+\vec{l}}$
Gesamtdrehimpuls}
\rput[r](0.8,0.6407812){\color{gdarkgray}Multiplizität $2s+1$}
\end{pspicture} 
  \caption{Notation der Atom-Zustände.}
\end{figure}

Die Wahrscheinlichkeit ist proportional zum Dipolmatrixelement
\begin{align*}
\text{Wsk} \propto \abs{\bra{{}^3S_1}\op{d}\op{E}\ket{{}^3P_2}}^2 =
\abs{\int\dV \Psi_{{}^3S_1}^*\op{r}\Psi_{{}^3P_2}}^2 \neq 0,
\end{align*}
d.h. der Übergang ist erlaubt. Betrachten wir hingegen den Übergang von
${}^3P_2$ zu ${}^3P_0$, dann ist das Integral antisymmetrisch,
\begin{align*}
\bra{{}^3P_0}\op{r}\ket{{}^3P_2} = 0,
\end{align*} 
d.h. ein Übergang aufgrund elektrischer Dipolstrahlung ist nicht erlaubt.\bsphere
\end{bspn}

\subsubsection{Paritätserhaltung bei Reaktionen}

Da die Parität keine kontinuierliche Transformation wie bspw. eine Drehung ist,
ist die Parität eine multiplikative Erhaltungsgröße, d.h. $\prod_i P_i$ ist
erhalten und nicht wie bspw. $\sum_i l_i$ wie bei der Drehung. 

Betrachte eine chemische Reaktion
\begin{align*}
a+b\mapsto c+d.
\end{align*}
Aufgrund der multiplikativen Erhaltung gilt für die Reaktion,
\begin{align*}
P_aP_b\cdot(-1)^l = P_cP_d\cdot(-1)^{l'},
\end{align*}
wobei $P_i$ die Eigenparität des Reaktionspartners $i$ und $l$, $l'$ den
Bahndrehimpuls vor bzw. nach der Reaktion beschreiben.

\subsubsection{Paritätsverletzung bei Schwacher Wechselwirkung}
Die physikalischen Gesetze, welche die starke und elektromagnetische Wechselwirkung
 beschreiben, sind invariant unter einer räumlichen Spiegelung des betrachteten
  Systems, dies wurde in der ersten Hälfte des 20. Jhd. in einer Vielzahl von
  Experimenten bestätigt. Jedoch äußerten T.D. Lee und C.N. Yang 1956
  Bedenken, dass die Paritätsinvarianz auch
  für die schwache Wechselwirkung gilt und schlugen Experimente zur Überprüfung
  vor.

1957 wurde im Experiment von C.S. Wu die Paritätsverletzung der Schwachen
Wechselwirkung nachgewiesen. Dabei wurde der $\beta$-Zerfall von ${}^{60}Co$
untersucht,
\begin{align*}
n\mapsto p^+ + e^- + \overline{\nu}_e.
\end{align*}
Es wurde festgestellt, dass es bei der Emission der Elektronen relativ zur
Kernspinachse eine Vorzugsrichtung gibt. D.h. die schwache Wechselwirkung
koppelt nur an ``linkshändige'' Elektronen (z.B. Spin $\downarrow$).

\sfigure[H]%
	{1-WuExperimentAufbau.pdf}%
	{G. Otter, R. Honecker, Atome –Moleküle –Kerne: Band II, Molekül-und Kernphysik, S. 378}%
	{Experimentelle Anordnung des Wu-Experiments. b) Zählraten der
	$\gamma$-Detektroen. c) relative $\beta^-$-Zählraten für zwei entgegengesetzte
 $\vec{B}_0$-Richtungen.}

\subsection{Zeitspiegelung}

Die Zeitspiegelung ist eine weitere diskrete Symmetrieoperation, die den
Ablauf der Zeit umkehrt.
\begin{align*}
T:\; & t\mapsto -t, && \vec{x}\mapsto \vec{x},\\
&\vec{p}\mapsto -\vec{p},&&\vec{L}\mapsto -\vec{L},\\
&\vec{E}\mapsto \vec{E},&&\vec{B}\mapsto -\vec{B}.
\end{align*}

Die Newtonschen Bewegungsgleichungen der klassischen Mechanik enthalten nur 2.
Ableitungen nach der Zeit. Daher ist die klassische Mechanik invariant unter
Zeitspiegelung so lange keine Dämpfung auftritt.

Die Maxwell Gleichungen enthalten ebenfalls nur 2. Ableitungen nach der Zeit
und sind somit ebenfalls invariant.

Wir wollen nun den Zeitspiegelungsoperator in der Quantenmechanik ``naiv'' durch
\begin{align*}
\hat{T}: \Psi(x,t) \mapsto \Psi(x,-t),
\end{align*}
definieren und die Schrödingergleichung betrachten.
\begin{align*}
i\hbar \frac{\partial \hat{T}\Psi(x,t)}{\partial t} \overset{!}{=}
\hat{H}\hat{T}\Psi(x,t)
\Leftrightarrow
-i\hbar \frac{\partial}{\partial (-t)}\Psi(x,-t) = \hat{H}\Psi(x,-t).
\end{align*}
Wir sehen, dass die zeitgespiegelte Wellenfunktion $\hat{T}\Psi$ die
Schrödingergleichung \textit{nicht} erfüllt. Die korrekte Definition des
Zeitspiegelungsoperators ist daher,
\begin{align*}
\hat{T}: \Psi(x,t)\mapsto \Psi^*(x,-t).
\end{align*}
Nun erfüllt $\hat{T}\Psi$ die Schrödingergleichung. Aufgrund der komplexen
Konjugation ist $\hat{T}$ kein unitärer Operator und besitzt weder 
Eigenwerte noch Eigenfunktionen!

\begin{bemn}[Bemerkungen.]
\begin{enumerate}[label=\arabic{*}.)]
  \item 
Es ist nicht möglich, $\hat{T}$ nur durch einen Wechsel des Bezugssystems zu
realisieren.
\item Experimentell wurde $\hat{T}$ durch Reversibilitätsexperimente getestet.
Bei der schwachen Wechselwirkung ($K^0$-Zerfall) wurden Hinweise auf eine
$\hat{T}$-Verletzung gefunden. $\hat{T}$ ist also keine universelle
Symmetrie.\maphere
\end{enumerate}
\end{bemn}

\subsection{Ladungsumkehr}
Die innere Symmetrie basiert auf Konventionen. Z.B. ist die Ladung des
Elektrons negativ definiert, während die des Protons positiv definiert ist. Die
\emph{Ladungsumkehr} $\hat{C}$ beschreibt die Umkehrung dieser Konvention:
\begin{align*}
\hat{C}\ket{\underbrace{m,\vec{p},s,m_s}_{\text{\tiny
äußere,}},\underbrace{q,B,L,S\ldots}_{\text{\tiny innere Eigenschaften}}}
= \underbrace{e^{i\phi}}_{\text{\tiny
Phase}}\ket{m,\vec{p},s,m_s,-q,-B,-L,-S,\ldots}
\end{align*}
$q$ steht hier für die Ladung bezüglich der elektromagnetischen Wechselwirkung,
während $B$ und $S$ Ladungen bezüglich der straken Wechselwirkung und $L$ Ladung
bezüglich der schwachen Wechselwirkung sind. Man spricht hier auch von
Farbladungen.

Die Ladungsumkehr spiegelt sowohl elektrische als auch magnetische Felder,
\begin{align*}
\vec{E}\mapsto -\vec{E},\qquad \vec{B} \mapsto -\vec{B}.
\end{align*}

\begin{bspn}
Betrachten wir ein linkshändiges Neutrino $\nu_L$ unter Ladungsumkehr
\begin{align*}
\hat{C}\ket{\nu_L} = e^{i\phi}\ket{\overline{\nu}_L}, 
\end{align*}
ergibt sich ein linkshändiges Antineutrino. Es existieren jedoch keine
linkshändigen Antineutrinos bzw. sie koppeln nicht mit der schwachen
Wechselwirkung. Es findet also eine vollständige Symmetriebrechnung bei der
schwachen Wechselwirkung statt. Für $\hat{C}\hat{P}$ gilt
\begin{align*}
\hat{C}\hat{P}\ket{\nu_L} = e^{i\phi'}\ket{\overline{\nu}_R}.
\end{align*}
$\overline{\nu}_R$ existiert und koppelt mit der Schwachen Wechselwirkung.
Jedoch wird auch  $\hat{C}\hat{P}$ beim $K^0$-Zefall verletzt und ist
daher ebenso keine universelle Symmetrie.\bsphere
\end{bspn}

Für neutrale Teilchen gilt,
\begin{align*}
\hat{C}\ket{m,\vec{p},s,m_s,q=0,B=0,\ldots} =
e^{i\phi}\ket{m,\vec{p},s,m_s,q=0,B=0,\ldots}
\end{align*}
Es muss daher gelten $\hat{C}^2 = \Id$ und daher sind die Eigenwerte von
$\hat{C}$ gerade $e^{i\phi}=\pm 1$. Wir erhalten somit gerade und
ungerade Ladungsparität.
\begin{bspn}
Da $\op{C}\vec{E} = -\vec{E}$ und $\op{C}\vec{B} = -\vec{B}$ besitzt
das Photon eine ungerade Ladungsparität.\bsphere
\end{bspn}
Wir erhalten dadurch Auswahlregeln, die uns bei Aussagen über Zerfallsprozesse
unterstützen.
\begin{bspn}
Ein neutrales $\pi^0$ zerfällt durch Emission von 2 Photonen,
\begin{align*}
&\pi^0 \to 2\gamma,\\
&\hat{C}\ket{\pi^0} = +\ket{\pi^0}.
\end{align*}
Die Emission von einem oder drei Photonen ist daher verboten.\bsphere
\end{bspn}

\subsection{Zusammenfassung diskrete Symmetrien}
\begin{table}[H]
\centering
\begin{tabular}[c]{l|lll}
 & $\hat{P}$ & $\hat{T}$ & $\hat{C}$\\\hline
$\vec{x}$ & -$\vec{x}$ & $\vec{x}$ & $\vec{x}$\\
$\vec{p}$ & -$\vec{p}$ & -$\vec{p}$ & $\vec{p}$\\
$\vec{L}$ & $\vec{L}$ & -$\vec{L}$ & $\vec{L}$\\
$\vec{E}$ & -$\vec{E}$ & $\vec{E}$ & -$\vec{E}$\\
$\vec{B}$ & $\vec{B}$ & -$\vec{B}$ & -$\vec{B}$\\
$\vec{q}$ & $\vec{q}$ & $\vec{q}$ & -$\vec{q}$  
\end{tabular}
\end{table}

Bisher wurde die $\hat{C}\hat{P}\hat{T}$ Symmetrie als \textit{grundlegende
Symmetrie} identifiziert. Es ist kein Experiment bekannt bei dem eine
 $\hat{C}\hat{P}\hat{T}$ Symmetriebrechnung einer Wechselwirkung nachgewiesen
werden kann. D.h. Teilchen und Antiteilchen haben gleiche Masse, gleiche
Lebensdauern und gleich starke aber entgegengesetzte magnetische Momente bzw.
elektrische Ladungen.
